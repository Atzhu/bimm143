% Options for packages loaded elsewhere
\PassOptionsToPackage{unicode}{hyperref}
\PassOptionsToPackage{hyphens}{url}
%
\documentclass[
]{article}
\usepackage{amsmath,amssymb}
\usepackage{iftex}
\ifPDFTeX
  \usepackage[T1]{fontenc}
  \usepackage[utf8]{inputenc}
  \usepackage{textcomp} % provide euro and other symbols
\else % if luatex or xetex
  \usepackage{unicode-math} % this also loads fontspec
  \defaultfontfeatures{Scale=MatchLowercase}
  \defaultfontfeatures[\rmfamily]{Ligatures=TeX,Scale=1}
\fi
\usepackage{lmodern}
\ifPDFTeX\else
  % xetex/luatex font selection
\fi
% Use upquote if available, for straight quotes in verbatim environments
\IfFileExists{upquote.sty}{\usepackage{upquote}}{}
\IfFileExists{microtype.sty}{% use microtype if available
  \usepackage[]{microtype}
  \UseMicrotypeSet[protrusion]{basicmath} % disable protrusion for tt fonts
}{}
\makeatletter
\@ifundefined{KOMAClassName}{% if non-KOMA class
  \IfFileExists{parskip.sty}{%
    \usepackage{parskip}
  }{% else
    \setlength{\parindent}{0pt}
    \setlength{\parskip}{6pt plus 2pt minus 1pt}}
}{% if KOMA class
  \KOMAoptions{parskip=half}}
\makeatother
\usepackage{xcolor}
\usepackage[margin=1in]{geometry}
\usepackage{color}
\usepackage{fancyvrb}
\newcommand{\VerbBar}{|}
\newcommand{\VERB}{\Verb[commandchars=\\\{\}]}
\DefineVerbatimEnvironment{Highlighting}{Verbatim}{commandchars=\\\{\}}
% Add ',fontsize=\small' for more characters per line
\usepackage{framed}
\definecolor{shadecolor}{RGB}{248,248,248}
\newenvironment{Shaded}{\begin{snugshade}}{\end{snugshade}}
\newcommand{\AlertTok}[1]{\textcolor[rgb]{0.94,0.16,0.16}{#1}}
\newcommand{\AnnotationTok}[1]{\textcolor[rgb]{0.56,0.35,0.01}{\textbf{\textit{#1}}}}
\newcommand{\AttributeTok}[1]{\textcolor[rgb]{0.13,0.29,0.53}{#1}}
\newcommand{\BaseNTok}[1]{\textcolor[rgb]{0.00,0.00,0.81}{#1}}
\newcommand{\BuiltInTok}[1]{#1}
\newcommand{\CharTok}[1]{\textcolor[rgb]{0.31,0.60,0.02}{#1}}
\newcommand{\CommentTok}[1]{\textcolor[rgb]{0.56,0.35,0.01}{\textit{#1}}}
\newcommand{\CommentVarTok}[1]{\textcolor[rgb]{0.56,0.35,0.01}{\textbf{\textit{#1}}}}
\newcommand{\ConstantTok}[1]{\textcolor[rgb]{0.56,0.35,0.01}{#1}}
\newcommand{\ControlFlowTok}[1]{\textcolor[rgb]{0.13,0.29,0.53}{\textbf{#1}}}
\newcommand{\DataTypeTok}[1]{\textcolor[rgb]{0.13,0.29,0.53}{#1}}
\newcommand{\DecValTok}[1]{\textcolor[rgb]{0.00,0.00,0.81}{#1}}
\newcommand{\DocumentationTok}[1]{\textcolor[rgb]{0.56,0.35,0.01}{\textbf{\textit{#1}}}}
\newcommand{\ErrorTok}[1]{\textcolor[rgb]{0.64,0.00,0.00}{\textbf{#1}}}
\newcommand{\ExtensionTok}[1]{#1}
\newcommand{\FloatTok}[1]{\textcolor[rgb]{0.00,0.00,0.81}{#1}}
\newcommand{\FunctionTok}[1]{\textcolor[rgb]{0.13,0.29,0.53}{\textbf{#1}}}
\newcommand{\ImportTok}[1]{#1}
\newcommand{\InformationTok}[1]{\textcolor[rgb]{0.56,0.35,0.01}{\textbf{\textit{#1}}}}
\newcommand{\KeywordTok}[1]{\textcolor[rgb]{0.13,0.29,0.53}{\textbf{#1}}}
\newcommand{\NormalTok}[1]{#1}
\newcommand{\OperatorTok}[1]{\textcolor[rgb]{0.81,0.36,0.00}{\textbf{#1}}}
\newcommand{\OtherTok}[1]{\textcolor[rgb]{0.56,0.35,0.01}{#1}}
\newcommand{\PreprocessorTok}[1]{\textcolor[rgb]{0.56,0.35,0.01}{\textit{#1}}}
\newcommand{\RegionMarkerTok}[1]{#1}
\newcommand{\SpecialCharTok}[1]{\textcolor[rgb]{0.81,0.36,0.00}{\textbf{#1}}}
\newcommand{\SpecialStringTok}[1]{\textcolor[rgb]{0.31,0.60,0.02}{#1}}
\newcommand{\StringTok}[1]{\textcolor[rgb]{0.31,0.60,0.02}{#1}}
\newcommand{\VariableTok}[1]{\textcolor[rgb]{0.00,0.00,0.00}{#1}}
\newcommand{\VerbatimStringTok}[1]{\textcolor[rgb]{0.31,0.60,0.02}{#1}}
\newcommand{\WarningTok}[1]{\textcolor[rgb]{0.56,0.35,0.01}{\textbf{\textit{#1}}}}
\usepackage{longtable,booktabs,array}
\usepackage{calc} % for calculating minipage widths
% Correct order of tables after \paragraph or \subparagraph
\usepackage{etoolbox}
\makeatletter
\patchcmd\longtable{\par}{\if@noskipsec\mbox{}\fi\par}{}{}
\makeatother
% Allow footnotes in longtable head/foot
\IfFileExists{footnotehyper.sty}{\usepackage{footnotehyper}}{\usepackage{footnote}}
\makesavenoteenv{longtable}
\usepackage{graphicx}
\makeatletter
\def\maxwidth{\ifdim\Gin@nat@width>\linewidth\linewidth\else\Gin@nat@width\fi}
\def\maxheight{\ifdim\Gin@nat@height>\textheight\textheight\else\Gin@nat@height\fi}
\makeatother
% Scale images if necessary, so that they will not overflow the page
% margins by default, and it is still possible to overwrite the defaults
% using explicit options in \includegraphics[width, height, ...]{}
\setkeys{Gin}{width=\maxwidth,height=\maxheight,keepaspectratio}
% Set default figure placement to htbp
\makeatletter
\def\fps@figure{htbp}
\makeatother
\setlength{\emergencystretch}{3em} % prevent overfull lines
\providecommand{\tightlist}{%
  \setlength{\itemsep}{0pt}\setlength{\parskip}{0pt}}
\setcounter{secnumdepth}{-\maxdimen} % remove section numbering
\ifLuaTeX
  \usepackage{selnolig}  % disable illegal ligatures
\fi
\usepackage{bookmark}
\IfFileExists{xurl.sty}{\usepackage{xurl}}{} % add URL line breaks if available
\urlstyle{same}
\hypersetup{
  pdftitle={Halloween Mini Project},
  pdfauthor={Audrey Ting Zhu (A16898668)},
  hidelinks,
  pdfcreator={LaTeX via pandoc}}

\title{Halloween Mini Project}
\author{Audrey Ting Zhu (A16898668)}
\date{2024-11-03}

\begin{document}
\maketitle

\#\#Class 10: Halloween Mini-Project

\begin{Shaded}
\begin{Highlighting}[]
\NormalTok{candy\_file }\OtherTok{\textless{}{-}} \StringTok{"candy{-}data.csv"}

\NormalTok{candy }\OtherTok{=} \FunctionTok{read.csv}\NormalTok{(candy\_file,}\AttributeTok{row.names=}\DecValTok{1}\NormalTok{)}
\FunctionTok{head}\NormalTok{(candy)}
\end{Highlighting}
\end{Shaded}

\begin{verbatim}
##              chocolate fruity caramel peanutyalmondy nougat crispedricewafer
## 100 Grand            1      0       1              0      0                1
## 3 Musketeers         1      0       0              0      1                0
## One dime             0      0       0              0      0                0
## One quarter          0      0       0              0      0                0
## Air Heads            0      1       0              0      0                0
## Almond Joy           1      0       0              1      0                0
##              hard bar pluribus sugarpercent pricepercent winpercent
## 100 Grand       0   1        0        0.732        0.860   66.97173
## 3 Musketeers    0   1        0        0.604        0.511   67.60294
## One dime        0   0        0        0.011        0.116   32.26109
## One quarter     0   0        0        0.011        0.511   46.11650
## Air Heads       0   0        0        0.906        0.511   52.34146
## Almond Joy      0   1        0        0.465        0.767   50.34755
\end{verbatim}

\begin{quote}
Q1. How many different candy types are in this dataset? ANS: There are
85 candy types.
\end{quote}

\begin{Shaded}
\begin{Highlighting}[]
\FunctionTok{nrow}\NormalTok{(candy)}
\end{Highlighting}
\end{Shaded}

\begin{verbatim}
## [1] 85
\end{verbatim}

\begin{quote}
Q2. How many fruity candy types are in the dataset? ANS: There are 38
fruity candy types.
\end{quote}

\begin{Shaded}
\begin{Highlighting}[]
\FunctionTok{sum}\NormalTok{(candy}\SpecialCharTok{$}\NormalTok{fruity}\SpecialCharTok{==}\DecValTok{1}\NormalTok{)}
\end{Highlighting}
\end{Shaded}

\begin{verbatim}
## [1] 38
\end{verbatim}

\begin{quote}
Q3. What is your favorite candy in the dataset and what is it's
winpercent value? ANS: My favorite candy is Air Heads. The winpercentage
is 52.34146.
\end{quote}

\begin{Shaded}
\begin{Highlighting}[]
\NormalTok{candy[}\StringTok{"Air Heads"}\NormalTok{, ]}\SpecialCharTok{$}\NormalTok{winpercent}
\end{Highlighting}
\end{Shaded}

\begin{verbatim}
## [1] 52.34146
\end{verbatim}

\begin{quote}
Q4. What is the winpercent value for ``Kit Kat''? ANS:76.7686
\end{quote}

\begin{Shaded}
\begin{Highlighting}[]
\NormalTok{candy[}\StringTok{"Kit Kat"}\NormalTok{, ]}\SpecialCharTok{$}\NormalTok{winpercent}
\end{Highlighting}
\end{Shaded}

\begin{verbatim}
## [1] 76.7686
\end{verbatim}

\begin{quote}
Q5. What is the winpercent value for ``Tootsie Roll Snack Bars''?
ANS:49.6535
\end{quote}

\begin{Shaded}
\begin{Highlighting}[]
\NormalTok{candy[}\StringTok{"Tootsie Roll Snack Bars"}\NormalTok{, ]}\SpecialCharTok{$}\NormalTok{winpercent}
\end{Highlighting}
\end{Shaded}

\begin{verbatim}
## [1] 49.6535
\end{verbatim}

\begin{Shaded}
\begin{Highlighting}[]
\FunctionTok{library}\NormalTok{(}\StringTok{"skimr"}\NormalTok{)}
\FunctionTok{skim}\NormalTok{(candy)}
\end{Highlighting}
\end{Shaded}

\begin{longtable}[]{@{}ll@{}}
\caption{Data summary}\tabularnewline
\toprule\noalign{}
\endfirsthead
\endhead
\bottomrule\noalign{}
\endlastfoot
Name & candy \\
Number of rows & 85 \\
Number of columns & 12 \\
\_\_\_\_\_\_\_\_\_\_\_\_\_\_\_\_\_\_\_\_\_\_\_ & \\
Column type frequency: & \\
numeric & 12 \\
\_\_\_\_\_\_\_\_\_\_\_\_\_\_\_\_\_\_\_\_\_\_\_\_ & \\
Group variables & None \\
\end{longtable}

\textbf{Variable type: numeric}

\begin{longtable}[]{@{}
  >{\raggedright\arraybackslash}p{(\columnwidth - 20\tabcolsep) * \real{0.1910}}
  >{\raggedleft\arraybackslash}p{(\columnwidth - 20\tabcolsep) * \real{0.1124}}
  >{\raggedleft\arraybackslash}p{(\columnwidth - 20\tabcolsep) * \real{0.1573}}
  >{\raggedleft\arraybackslash}p{(\columnwidth - 20\tabcolsep) * \real{0.0674}}
  >{\raggedleft\arraybackslash}p{(\columnwidth - 20\tabcolsep) * \real{0.0674}}
  >{\raggedleft\arraybackslash}p{(\columnwidth - 20\tabcolsep) * \real{0.0674}}
  >{\raggedleft\arraybackslash}p{(\columnwidth - 20\tabcolsep) * \real{0.0674}}
  >{\raggedleft\arraybackslash}p{(\columnwidth - 20\tabcolsep) * \real{0.0674}}
  >{\raggedleft\arraybackslash}p{(\columnwidth - 20\tabcolsep) * \real{0.0674}}
  >{\raggedleft\arraybackslash}p{(\columnwidth - 20\tabcolsep) * \real{0.0674}}
  >{\raggedright\arraybackslash}p{(\columnwidth - 20\tabcolsep) * \real{0.0674}}@{}}
\toprule\noalign{}
\begin{minipage}[b]{\linewidth}\raggedright
skim\_variable
\end{minipage} & \begin{minipage}[b]{\linewidth}\raggedleft
n\_missing
\end{minipage} & \begin{minipage}[b]{\linewidth}\raggedleft
complete\_rate
\end{minipage} & \begin{minipage}[b]{\linewidth}\raggedleft
mean
\end{minipage} & \begin{minipage}[b]{\linewidth}\raggedleft
sd
\end{minipage} & \begin{minipage}[b]{\linewidth}\raggedleft
p0
\end{minipage} & \begin{minipage}[b]{\linewidth}\raggedleft
p25
\end{minipage} & \begin{minipage}[b]{\linewidth}\raggedleft
p50
\end{minipage} & \begin{minipage}[b]{\linewidth}\raggedleft
p75
\end{minipage} & \begin{minipage}[b]{\linewidth}\raggedleft
p100
\end{minipage} & \begin{minipage}[b]{\linewidth}\raggedright
hist
\end{minipage} \\
\midrule\noalign{}
\endhead
\bottomrule\noalign{}
\endlastfoot
chocolate & 0 & 1 & 0.44 & 0.50 & 0.00 & 0.00 & 0.00 & 1.00 & 1.00 &
▇▁▁▁▆ \\
fruity & 0 & 1 & 0.45 & 0.50 & 0.00 & 0.00 & 0.00 & 1.00 & 1.00 &
▇▁▁▁▆ \\
caramel & 0 & 1 & 0.16 & 0.37 & 0.00 & 0.00 & 0.00 & 0.00 & 1.00 &
▇▁▁▁▂ \\
peanutyalmondy & 0 & 1 & 0.16 & 0.37 & 0.00 & 0.00 & 0.00 & 0.00 & 1.00
& ▇▁▁▁▂ \\
nougat & 0 & 1 & 0.08 & 0.28 & 0.00 & 0.00 & 0.00 & 0.00 & 1.00 &
▇▁▁▁▁ \\
crispedricewafer & 0 & 1 & 0.08 & 0.28 & 0.00 & 0.00 & 0.00 & 0.00 &
1.00 & ▇▁▁▁▁ \\
hard & 0 & 1 & 0.18 & 0.38 & 0.00 & 0.00 & 0.00 & 0.00 & 1.00 & ▇▁▁▁▂ \\
bar & 0 & 1 & 0.25 & 0.43 & 0.00 & 0.00 & 0.00 & 0.00 & 1.00 & ▇▁▁▁▂ \\
pluribus & 0 & 1 & 0.52 & 0.50 & 0.00 & 0.00 & 1.00 & 1.00 & 1.00 &
▇▁▁▁▇ \\
sugarpercent & 0 & 1 & 0.48 & 0.28 & 0.01 & 0.22 & 0.47 & 0.73 & 0.99 &
▇▇▇▇▆ \\
pricepercent & 0 & 1 & 0.47 & 0.29 & 0.01 & 0.26 & 0.47 & 0.65 & 0.98 &
▇▇▇▇▆ \\
winpercent & 0 & 1 & 50.32 & 14.71 & 22.45 & 39.14 & 47.83 & 59.86 &
84.18 & ▃▇▆▅▂ \\
\end{longtable}

\begin{quote}
Q6. Is there any variable/column that looks to be on a different scale
to the majority of the other columns in the dataset?
\end{quote}

\begin{quote}
ANS: Winpercent is in a different range as it is a percentage score
ranging up to 100. The other variables scale up to 1.
\end{quote}

\begin{quote}
Q7. What do you think a zero and one represent for the candy\$chocolate
column?
\end{quote}

\begin{quote}
ANS: 0 means that there is no chocolate in the candy composition. 1
means that there is chocolate in the candy.
\end{quote}

\begin{quote}
Q8. Plot a histogram of winpercent values
\end{quote}

\begin{Shaded}
\begin{Highlighting}[]
\FunctionTok{library}\NormalTok{(ggplot2)}

\FunctionTok{ggplot}\NormalTok{(candy, }\FunctionTok{aes}\NormalTok{(}\AttributeTok{x=}\NormalTok{winpercent))}\SpecialCharTok{+}\FunctionTok{geom\_histogram}\NormalTok{(}\AttributeTok{binwidth=}\DecValTok{5}\NormalTok{)}\SpecialCharTok{+}\FunctionTok{labs}\NormalTok{(}\AttributeTok{title=}\StringTok{"Winpercentages for Candies"}\NormalTok{, }\AttributeTok{x=}\StringTok{"Winpercent"}\NormalTok{)}
\end{Highlighting}
\end{Shaded}

\includegraphics{Halloween-Mini-Project_files/figure-latex/unnamed-chunk-8-1.pdf}

\begin{quote}
Q9. Is the distribution of winpercent values symmetrical?
\end{quote}

\begin{quote}
ANS: The distribution of winpercent values are not 100 percent
symmetrical, but it is roughly.
\end{quote}

\begin{quote}
Q10. Is the center of the distribution above or below 50\%?
\end{quote}

\begin{quote}
ANS: Depends if you look at mean or median. The mean is slightly above
50 but the median is 47.82, which is below 50\%.
\end{quote}

\begin{Shaded}
\begin{Highlighting}[]
\FunctionTok{mean}\NormalTok{(candy}\SpecialCharTok{$}\NormalTok{winpercent)}
\end{Highlighting}
\end{Shaded}

\begin{verbatim}
## [1] 50.31676
\end{verbatim}

\begin{Shaded}
\begin{Highlighting}[]
\FunctionTok{median}\NormalTok{(candy}\SpecialCharTok{$}\NormalTok{winpercent)}
\end{Highlighting}
\end{Shaded}

\begin{verbatim}
## [1] 47.82975
\end{verbatim}

\begin{quote}
Q11. On average is chocolate candy higher or lower ranked than fruit
candy?
\end{quote}

\begin{quote}
ANS: Chocolate candy has an higher score.
\end{quote}

\begin{Shaded}
\begin{Highlighting}[]
\FunctionTok{mean}\NormalTok{(candy}\SpecialCharTok{$}\NormalTok{winpercent[}\FunctionTok{as.logical}\NormalTok{(candy}\SpecialCharTok{$}\NormalTok{chocolate)])}
\end{Highlighting}
\end{Shaded}

\begin{verbatim}
## [1] 60.92153
\end{verbatim}

\begin{Shaded}
\begin{Highlighting}[]
\FunctionTok{mean}\NormalTok{(candy}\SpecialCharTok{$}\NormalTok{winpercent[}\FunctionTok{as.logical}\NormalTok{(candy}\SpecialCharTok{$}\NormalTok{fruity)])}
\end{Highlighting}
\end{Shaded}

\begin{verbatim}
## [1] 44.11974
\end{verbatim}

\begin{quote}
Q12. Is this difference statistically significant?
\end{quote}

\begin{Shaded}
\begin{Highlighting}[]
\FunctionTok{t.test}\NormalTok{(candy}\SpecialCharTok{$}\NormalTok{winpercent[}\FunctionTok{as.logical}\NormalTok{(candy}\SpecialCharTok{$}\NormalTok{chocolate)],candy}\SpecialCharTok{$}\NormalTok{winpercent[}\FunctionTok{as.logical}\NormalTok{(candy}\SpecialCharTok{$}\NormalTok{fruity)])}
\end{Highlighting}
\end{Shaded}

\begin{verbatim}
## 
##  Welch Two Sample t-test
## 
## data:  candy$winpercent[as.logical(candy$chocolate)] and candy$winpercent[as.logical(candy$fruity)]
## t = 6.2582, df = 68.882, p-value = 2.871e-08
## alternative hypothesis: true difference in means is not equal to 0
## 95 percent confidence interval:
##  11.44563 22.15795
## sample estimates:
## mean of x mean of y 
##  60.92153  44.11974
\end{verbatim}

\begin{quote}
ANS: P-value is less than 0.05. It is pretty significant.
\end{quote}

\begin{quote}
Q13. What are the five least liked candy types in this set?
\end{quote}

\begin{Shaded}
\begin{Highlighting}[]
\FunctionTok{head}\NormalTok{(candy[}\FunctionTok{order}\NormalTok{(}\SpecialCharTok{{-}}\NormalTok{candy}\SpecialCharTok{$}\NormalTok{winpercent),], }\AttributeTok{n=}\DecValTok{5}\NormalTok{)}
\end{Highlighting}
\end{Shaded}

\begin{verbatim}
##                           chocolate fruity caramel peanutyalmondy nougat
## Reese's Peanut Butter cup         1      0       0              1      0
## Reese's Miniatures                1      0       0              1      0
## Twix                              1      0       1              0      0
## Kit Kat                           1      0       0              0      0
## Snickers                          1      0       1              1      1
##                           crispedricewafer hard bar pluribus sugarpercent
## Reese's Peanut Butter cup                0    0   0        0        0.720
## Reese's Miniatures                       0    0   0        0        0.034
## Twix                                     1    0   1        0        0.546
## Kit Kat                                  1    0   1        0        0.313
## Snickers                                 0    0   1        0        0.546
##                           pricepercent winpercent
## Reese's Peanut Butter cup        0.651   84.18029
## Reese's Miniatures               0.279   81.86626
## Twix                             0.906   81.64291
## Kit Kat                          0.511   76.76860
## Snickers                         0.651   76.67378
\end{verbatim}

\begin{quote}
Q14. What are the top 5 all time favorite candy types out of this set?
ANS: The benefit of dplyr is that it has much simpler and readable
syntax. However, you also need to download a package to make it work.
\end{quote}

\begin{Shaded}
\begin{Highlighting}[]
\FunctionTok{library}\NormalTok{(}\StringTok{"dplyr"}\NormalTok{)}
\end{Highlighting}
\end{Shaded}

\begin{verbatim}
## 
## 载入程序包:'dplyr'
\end{verbatim}

\begin{verbatim}
## The following objects are masked from 'package:stats':
## 
##     filter, lag
\end{verbatim}

\begin{verbatim}
## The following objects are masked from 'package:base':
## 
##     intersect, setdiff, setequal, union
\end{verbatim}

\begin{Shaded}
\begin{Highlighting}[]
\NormalTok{candy }\SpecialCharTok{\%\textgreater{}\%} \FunctionTok{arrange}\NormalTok{(winpercent) }\SpecialCharTok{\%\textgreater{}\%} \FunctionTok{head}\NormalTok{(}\DecValTok{5}\NormalTok{)}
\end{Highlighting}
\end{Shaded}

\begin{verbatim}
##                    chocolate fruity caramel peanutyalmondy nougat
## Nik L Nip                  0      1       0              0      0
## Boston Baked Beans         0      0       0              1      0
## Chiclets                   0      1       0              0      0
## Super Bubble               0      1       0              0      0
## Jawbusters                 0      1       0              0      0
##                    crispedricewafer hard bar pluribus sugarpercent pricepercent
## Nik L Nip                         0    0   0        1        0.197        0.976
## Boston Baked Beans                0    0   0        1        0.313        0.511
## Chiclets                          0    0   0        1        0.046        0.325
## Super Bubble                      0    0   0        0        0.162        0.116
## Jawbusters                        0    1   0        1        0.093        0.511
##                    winpercent
## Nik L Nip            22.44534
## Boston Baked Beans   23.41782
## Chiclets             24.52499
## Super Bubble         27.30386
## Jawbusters           28.12744
\end{verbatim}

\begin{quote}
Q15. Make a first barplot of candy ranking based on winpercent values.
\end{quote}

\begin{Shaded}
\begin{Highlighting}[]
\FunctionTok{library}\NormalTok{(ggplot2)}

\FunctionTok{ggplot}\NormalTok{(candy) }\SpecialCharTok{+} 
  \FunctionTok{aes}\NormalTok{(winpercent, }\FunctionTok{rownames}\NormalTok{(candy)) }\SpecialCharTok{+}
  \FunctionTok{geom\_bar}\NormalTok{(}\AttributeTok{stat=}\StringTok{"identity"}\NormalTok{)}
\end{Highlighting}
\end{Shaded}

\includegraphics{Halloween-Mini-Project_files/figure-latex/unnamed-chunk-15-1.pdf}

\begin{quote}
Q16. This is quite ugly, use the reorder() function to get the bars
sorted by winpercent?
\end{quote}

\begin{Shaded}
\begin{Highlighting}[]
\FunctionTok{library}\NormalTok{(ggplot2)}

\FunctionTok{ggplot}\NormalTok{(candy) }\SpecialCharTok{+} 
  \FunctionTok{aes}\NormalTok{(winpercent, }\FunctionTok{reorder}\NormalTok{(}\FunctionTok{rownames}\NormalTok{(candy),winpercent)) }\SpecialCharTok{+}
  \FunctionTok{geom\_bar}\NormalTok{(}\AttributeTok{stat=}\StringTok{"identity"}\NormalTok{)}
\end{Highlighting}
\end{Shaded}

\includegraphics{Halloween-Mini-Project_files/figure-latex/unnamed-chunk-16-1.pdf}

\begin{Shaded}
\begin{Highlighting}[]
\NormalTok{my\_cols}\OtherTok{=}\FunctionTok{rep}\NormalTok{(}\StringTok{"black"}\NormalTok{, }\FunctionTok{nrow}\NormalTok{(candy))}
\NormalTok{my\_cols[}\FunctionTok{as.logical}\NormalTok{(candy}\SpecialCharTok{$}\NormalTok{chocolate)] }\OtherTok{=} \StringTok{"chocolate"}
\NormalTok{my\_cols[}\FunctionTok{as.logical}\NormalTok{(candy}\SpecialCharTok{$}\NormalTok{bar)] }\OtherTok{=} \StringTok{"brown"}
\NormalTok{my\_cols[}\FunctionTok{as.logical}\NormalTok{(candy}\SpecialCharTok{$}\NormalTok{fruity)] }\OtherTok{=} \StringTok{"pink"}

\FunctionTok{ggplot}\NormalTok{(candy) }\SpecialCharTok{+} 
  \FunctionTok{aes}\NormalTok{(winpercent, }\FunctionTok{reorder}\NormalTok{(}\FunctionTok{rownames}\NormalTok{(candy),winpercent)) }\SpecialCharTok{+}
  \FunctionTok{geom\_col}\NormalTok{(}\AttributeTok{fill=}\NormalTok{my\_cols) }
\end{Highlighting}
\end{Shaded}

\includegraphics{Halloween-Mini-Project_files/figure-latex/unnamed-chunk-17-1.pdf}
\textgreater Q17. What is the worst ranked chocolate candy?
\textgreater ANS: Apparently people dont sixlets. \textgreater Q18. What
is the best ranked fruity candy? \textgreater ANS: People like
Starburst.

\begin{Shaded}
\begin{Highlighting}[]
\FunctionTok{library}\NormalTok{(}\StringTok{"ggrepel"}\NormalTok{)}
\end{Highlighting}
\end{Shaded}

\begin{Shaded}
\begin{Highlighting}[]
\FunctionTok{ggplot}\NormalTok{(candy) }\SpecialCharTok{+}
  \FunctionTok{aes}\NormalTok{(winpercent, pricepercent, }\AttributeTok{label=}\FunctionTok{rownames}\NormalTok{(candy)) }\SpecialCharTok{+}
  \FunctionTok{geom\_point}\NormalTok{(}\AttributeTok{col=}\NormalTok{my\_cols) }\SpecialCharTok{+} 
  \FunctionTok{geom\_text\_repel}\NormalTok{(}\AttributeTok{col=}\NormalTok{my\_cols, }\AttributeTok{size=}\FloatTok{3.3}\NormalTok{, }\AttributeTok{max.overlaps =} \DecValTok{5}\NormalTok{)}
\end{Highlighting}
\end{Shaded}

\begin{verbatim}
## Warning: ggrepel: 54 unlabeled data points (too many overlaps). Consider
## increasing max.overlaps
\end{verbatim}

\includegraphics{Halloween-Mini-Project_files/figure-latex/unnamed-chunk-19-1.pdf}
\textgreater Q19. Which candy type is the highest ranked in terms of
winpercent for the least money - i.e.~offers the most bang for your
buck? ANS: Reeses Minatures

\begin{quote}
Q20. What are the top 5 most expensive candy types in the dataset and of
these which is the least popular?
\end{quote}

\begin{quote}
ANS: Nik L Lip manages to be expensive and unliked the most.
\end{quote}

\begin{Shaded}
\begin{Highlighting}[]
\NormalTok{ord }\OtherTok{\textless{}{-}} \FunctionTok{order}\NormalTok{(candy}\SpecialCharTok{$}\NormalTok{pricepercent, }\AttributeTok{decreasing =} \ConstantTok{TRUE}\NormalTok{)}
\FunctionTok{head}\NormalTok{( candy[ord,}\FunctionTok{c}\NormalTok{(}\DecValTok{11}\NormalTok{,}\DecValTok{12}\NormalTok{)], }\AttributeTok{n=}\DecValTok{5}\NormalTok{ )}
\end{Highlighting}
\end{Shaded}

\begin{verbatim}
##                          pricepercent winpercent
## Nik L Nip                       0.976   22.44534
## Nestle Smarties                 0.976   37.88719
## Ring pop                        0.965   35.29076
## Hershey's Krackel               0.918   62.28448
## Hershey's Milk Chocolate        0.918   56.49050
\end{verbatim}

\begin{quote}
Q21. Make a barplot again with geom\_col() this time using pricepercent
and then improve this step by step, first ordering the x-axis by value
and finally making a so called ``dot chat'' or ``lollipop'' chart by
swapping geom\_col() for geom\_point() + geom\_segment().
\end{quote}

\begin{Shaded}
\begin{Highlighting}[]
\CommentTok{\# Make a lollipop chart of pricepercent}
\FunctionTok{ggplot}\NormalTok{(candy) }\SpecialCharTok{+}
  \FunctionTok{aes}\NormalTok{(pricepercent, }\FunctionTok{reorder}\NormalTok{(}\FunctionTok{rownames}\NormalTok{(candy), pricepercent)) }\SpecialCharTok{+}
  \FunctionTok{geom\_segment}\NormalTok{(}\FunctionTok{aes}\NormalTok{(}\AttributeTok{yend =} \FunctionTok{reorder}\NormalTok{(}\FunctionTok{rownames}\NormalTok{(candy), pricepercent), }
                   \AttributeTok{xend =} \DecValTok{0}\NormalTok{), }\AttributeTok{col=}\StringTok{"gray40"}\NormalTok{) }\SpecialCharTok{+}
    \FunctionTok{geom\_point}\NormalTok{()}
\end{Highlighting}
\end{Shaded}

\includegraphics{Halloween-Mini-Project_files/figure-latex/unnamed-chunk-21-1.pdf}

\begin{Shaded}
\begin{Highlighting}[]
\FunctionTok{library}\NormalTok{(corrplot)}
\end{Highlighting}
\end{Shaded}

\begin{verbatim}
## corrplot 0.95 loaded
\end{verbatim}

\begin{Shaded}
\begin{Highlighting}[]
\NormalTok{cij }\OtherTok{\textless{}{-}} \FunctionTok{cor}\NormalTok{(candy)}
\FunctionTok{corrplot}\NormalTok{(cij)}
\end{Highlighting}
\end{Shaded}

\includegraphics{Halloween-Mini-Project_files/figure-latex/unnamed-chunk-22-1.pdf}
\textgreater Q22. Examining this plot what two variables are
anti-correlated (i.e.~have minus values)?

\begin{quote}
ANS: Chocolate and fruity are anti-corellated, which is a shame because
I like chocolate fruity candies.
\end{quote}

\begin{quote}
Q23. Similarly, what two variables are most positively correlated? Ans:
Chocolate and winpercent are correlated. Chocolate and bar are also
correlated.
\end{quote}

\begin{Shaded}
\begin{Highlighting}[]
\NormalTok{pca }\OtherTok{\textless{}{-}} \FunctionTok{prcomp}\NormalTok{(candy, }\AttributeTok{scale=}\ConstantTok{TRUE}\NormalTok{)}
\FunctionTok{summary}\NormalTok{(pca)}
\end{Highlighting}
\end{Shaded}

\begin{verbatim}
## Importance of components:
##                           PC1    PC2    PC3     PC4    PC5     PC6     PC7
## Standard deviation     2.0788 1.1378 1.1092 1.07533 0.9518 0.81923 0.81530
## Proportion of Variance 0.3601 0.1079 0.1025 0.09636 0.0755 0.05593 0.05539
## Cumulative Proportion  0.3601 0.4680 0.5705 0.66688 0.7424 0.79830 0.85369
##                            PC8     PC9    PC10    PC11    PC12
## Standard deviation     0.74530 0.67824 0.62349 0.43974 0.39760
## Proportion of Variance 0.04629 0.03833 0.03239 0.01611 0.01317
## Cumulative Proportion  0.89998 0.93832 0.97071 0.98683 1.00000
\end{verbatim}

\begin{Shaded}
\begin{Highlighting}[]
\NormalTok{pca}\SpecialCharTok{$}\NormalTok{rotation[,}\DecValTok{1}\NormalTok{]}
\end{Highlighting}
\end{Shaded}

\begin{verbatim}
##        chocolate           fruity          caramel   peanutyalmondy 
##       -0.4019466        0.3683883       -0.2299709       -0.2407155 
##           nougat crispedricewafer             hard              bar 
##       -0.2268102       -0.2215182        0.2111587       -0.3947433 
##         pluribus     sugarpercent     pricepercent       winpercent 
##        0.2600041       -0.1083088       -0.3207361       -0.3298035
\end{verbatim}

\begin{Shaded}
\begin{Highlighting}[]
\FunctionTok{plot}\NormalTok{(pca}\SpecialCharTok{$}\NormalTok{x[,}\DecValTok{1}\SpecialCharTok{:}\DecValTok{2}\NormalTok{],}\AttributeTok{col=}\NormalTok{my\_cols, }\AttributeTok{pch=}\DecValTok{16}\NormalTok{)}
\end{Highlighting}
\end{Shaded}

\includegraphics{Halloween-Mini-Project_files/figure-latex/unnamed-chunk-24-1.pdf}

\begin{Shaded}
\begin{Highlighting}[]
\CommentTok{\# Make a new data{-}frame with our PCA results and candy data}
\NormalTok{my\_data }\OtherTok{\textless{}{-}} \FunctionTok{cbind}\NormalTok{(candy, pca}\SpecialCharTok{$}\NormalTok{x[,}\DecValTok{1}\SpecialCharTok{:}\DecValTok{3}\NormalTok{])}
\end{Highlighting}
\end{Shaded}

\begin{Shaded}
\begin{Highlighting}[]
\NormalTok{p }\OtherTok{\textless{}{-}} \FunctionTok{ggplot}\NormalTok{(my\_data) }\SpecialCharTok{+} 
        \FunctionTok{aes}\NormalTok{(}\AttributeTok{x=}\NormalTok{PC1, }\AttributeTok{y=}\NormalTok{PC2, }
            \AttributeTok{size=}\NormalTok{winpercent}\SpecialCharTok{/}\DecValTok{100}\NormalTok{,  }
            \AttributeTok{text=}\FunctionTok{rownames}\NormalTok{(my\_data),}
            \AttributeTok{label=}\FunctionTok{rownames}\NormalTok{(my\_data)) }\SpecialCharTok{+}
        \FunctionTok{geom\_point}\NormalTok{(}\AttributeTok{col=}\NormalTok{my\_cols)}

\NormalTok{p}
\end{Highlighting}
\end{Shaded}

\includegraphics{Halloween-Mini-Project_files/figure-latex/unnamed-chunk-26-1.pdf}

\begin{Shaded}
\begin{Highlighting}[]
\FunctionTok{library}\NormalTok{(ggrepel)}

\NormalTok{p }\SpecialCharTok{+} \FunctionTok{geom\_text\_repel}\NormalTok{(}\AttributeTok{size=}\FloatTok{3.3}\NormalTok{, }\AttributeTok{col=}\NormalTok{my\_cols, }\AttributeTok{max.overlaps =} \DecValTok{7}\NormalTok{)  }\SpecialCharTok{+} 
  \FunctionTok{theme}\NormalTok{(}\AttributeTok{legend.position =} \StringTok{"none"}\NormalTok{) }\SpecialCharTok{+}
  \FunctionTok{labs}\NormalTok{(}\AttributeTok{title=}\StringTok{"Halloween Candy PCA Space"}\NormalTok{,}
       \AttributeTok{subtitle=}\StringTok{"Colored by type: chocolate bar (dark brown), chocolate other (light brown), fruity (red), other (black)"}\NormalTok{,}
       \AttributeTok{caption=}\StringTok{"Data from 538"}\NormalTok{)}
\end{Highlighting}
\end{Shaded}

\begin{verbatim}
## Warning: ggrepel: 44 unlabeled data points (too many overlaps). Consider
## increasing max.overlaps
\end{verbatim}

\includegraphics{Halloween-Mini-Project_files/figure-latex/unnamed-chunk-27-1.pdf}

\begin{Shaded}
\begin{Highlighting}[]
\StringTok{"library(plotly)}
\StringTok{ggplotly(p)"}
\end{Highlighting}
\end{Shaded}

\begin{verbatim}
## [1] "library(plotly)\nggplotly(p)"
\end{verbatim}

\begin{Shaded}
\begin{Highlighting}[]
\FunctionTok{par}\NormalTok{(}\AttributeTok{mar=}\FunctionTok{c}\NormalTok{(}\DecValTok{8}\NormalTok{,}\DecValTok{4}\NormalTok{,}\DecValTok{2}\NormalTok{,}\DecValTok{2}\NormalTok{))}
\FunctionTok{barplot}\NormalTok{(pca}\SpecialCharTok{$}\NormalTok{rotation[,}\DecValTok{1}\NormalTok{], }\AttributeTok{las=}\DecValTok{2}\NormalTok{, }\AttributeTok{ylab=}\StringTok{"PC1 Contribution"}\NormalTok{)}
\end{Highlighting}
\end{Shaded}

\includegraphics{Halloween-Mini-Project_files/figure-latex/unnamed-chunk-29-1.pdf}

\begin{quote}
Q24. What original variables are picked up strongly by PC1 in the
positive direction? Do these make sense to you? ANS:Fruity, hard and
pluribus are heavily picked up by PC1. This makes sense. Afterall, many
candies have the characteristics of being packed with multiple pieces in
a bag, being fruity, and being hard overlap together.
\end{quote}

\end{document}
